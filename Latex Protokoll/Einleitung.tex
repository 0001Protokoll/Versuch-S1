%\input{usepackage.tex}
%\begin{document}

\section{Einleitung}


Viele Eigenschaften von Molekülen lassen sich durch das Konzept der Quantenmechanik verstehen. Die Beschreibung und Berechnung dieser Eigenschaften ist analytisch nicht immer möglich, und numerisch erst durch die Entwicklung von leistungsstarken Computern erschwinglich geworden. Bei einer numerischen Lösung von quantenmechanischen Problemen begiebt sich der Wissenschaftler auf einen schmalen Grad aus optimalen Ergebnissen und geringsten Rechenaufwand, muss folglich also besonders interdiziplinär arbeiten. Solche Untersuchungen sind im wesentlichen Teilgebiet der theoretischen Physik und Chemie, haben jedoch als Möglichkeit der Vorraussage und Einschätzung den Weg in ein weites Feld der angewandter Wissenschaft wie Materialwissenschaft, organische und anorganische Chemie, Festkörperphysik sowie supramolekulare Chemie gefunden.


%\end{document}
