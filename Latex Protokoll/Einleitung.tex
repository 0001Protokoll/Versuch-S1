\documentclass[a4paper]{scrartcl}

\begin{document}

\section{Einleitung}
 Mit der Schrödingergleichung ist es möglich zum Beispiel Potentialenergieflächen zu berechnen, die einem Informationen zu den charakteristischen Schwingungen  geben. Jedoch steigt die Komplexität der Gleichung stark an, umso größer die Systeme werden. Eine Möglichkeit der Simplifizierung ist hier die Born-Oppenheimer-Näherung. Sie approximiert aufgrund der vielfach höheren Masse der Kerne im Vergleich zu den Elektronen, dass die Bewegungen der Kerne und die der Elektronen separierbar sind. Außerdem kann angenommen werden dass die kinetische Energie der Kerne gleich Null und somit die potentielle Energie konstant ist. Isotopeneffekte spielen aufgrund dieser Vereinfachung keine Rolle.
Durch Lösen der dieser Gleichungen kann nun das Potential bei unterschiedlichen Kern-Kern Abständen berechnet werden, wodurch man die Potentialenergieflächen erhält.





\end{document}
