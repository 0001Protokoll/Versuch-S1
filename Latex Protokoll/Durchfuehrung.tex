%\documentclass[12pt, letterpaper]{article}

 
%\begin{document}
\section{Durchführung}
Die Prozessierung erfolgt durch einen input file, welcher alle wichtigen Informationen der durchzuführenden Berechnung dem Computer übergibt. Dieser file beinhaltet im wesentlichen drei wichtige Blöcke. Der erste Befehlsblock am Anfang bezieht sich auf zugewiesene hardware Rechte, der folgende Block übergibt die Methode sowie weitere frei wählbare, vorimplementierte Optionen für die Berechnung (z.B die rekursive Berechnung für mehrere vordefinierte Bindungsabstände). Der letzte Block beinhaltet im wesentlichen die Eigenschaften des betrachteten quantenmechanischen Systems, wie die Gesamtladung, Spineigenschaften, sowie Atomtypen und Geometrien. Die Geometrien können sowohl in kartesischen Koordinaten, als auch in einer Z-Matrix angegeben sein. Die Darstellung in der Z-Matrix impliziert eine besonders große Vergleichbarkeit von geometrischen Eigenschaften wie Bindungslängen/- Winkeln und Torsionswinkeln im Molekül zu weiteren Referenzsystemen mit gleichem betrachteten Molekül.\\
\\
Zur Analyse wurden die zu berechnenden Daten aus dem output file manuell exportiert und weiter in -python- bearbeitet.

%\end{document}