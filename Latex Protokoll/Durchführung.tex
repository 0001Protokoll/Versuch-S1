%\documentclass[12pt, letterpaper]{article}

 
%\begin{document}
\section{Einführung} : \\
\\
Zur Bestimmung der gesuchten Energiewerte bezogen auf die Bindungslänge der Zweiatomigen Moleküle CO und HCl wurde mithilfe des Programms -Gaussian- die zeitunabhängige Schrödingergleichung gelöst. Im Allgemeinen löst -Gaussian- die Wellenfunktion im Rahmen der Born Oppenheimer Näherung genau für eine Kerngeometrie. Folglich wurde ein Intervall von Bindungslängen der betreffenden Moleküle gewählt, und jeweils für jede Geometrie ein Energiewert bestimmt, um representative Ergebnisse zu erhalten.\\	
\\
Berechnet wurden die Energien für beide Moleküle mit selbem Intervall an Bindungslängen in zwei unterschiedlichen Methoden (B3LYP \& CCSD). Allgemein bestehen die unterschiede der Methoden in den verwendeten Näherungen, um Elektron-Elektron WW zu beschreiben. Im wesentlichen beruhen die Näherung auf Dichtefunktionalen für Elektronen und der daraus resultierenden WW (DFT) oder auf Beschreibung wirkenden Kräfte auf den Elektronen durch ein externes Kraftfeld (HF). Relativistische Effekte der Elektronen können bei leichten Molekülen wie HCl und CO vernachlässigt werden. Der bei allen Berechnungen verwendete Basissatz () aus modifizierten Wasserstofforbitalen wurde, aufgrund geringer Anzahl an Basen <=> geringe Rechenzeit und auf die Komplexität der Basen bezogen gute Qualität, gewählt.\\
	
Durchführung : \\
\\
Die Prozessierung erfolgt durch einen input file, welcher alle wichtigen Informationen der durchzuführenden Berechnung dem Computer übergibt. Dieser file beinhaltet im wesentlichen drei wichtige Blöcke. Der erste Befehlsblock am Anfang bezieht sich auf zugewiesene hardware Rechte, der folgende Block übergibt die Methode sowie weitere frei wählbare, vorimplementierte Optionen für die Berechnung (z.B die rekursive Berechnung für mehrere vordefinierte Bindungsabstände). Der letzte Block beinhaltet im wesentlichen die Eigenschaften des betrachteten quantenmechanischen Systems, wie die Gesamtladung, Spineigenschaften, sowie Atomtypen und Geometrien. Die Geometrien können sowohl in kartesischen Koordinaten\\
\\
Zur Analyse wurden die zu berechnenden Daten aus dem output file manuell exportiert und weiter in -python- bearbeitet.

%\end{document}