\input{usepackage.tex}
\begin{document}

\section{Theoretische Grundlagen}

Die im Folgenden beschriebenen Systeme werden mit der zeitunabhängigen Schrödingergleichung beschrieben (Gleichung 1). 

\begin {equation}
\mathcal{H}\left(r,R\right)\psi\left(r,R\right)=E\psi\left(r,R\right)
\end {equation}

Jedoch steigt die Komplexität der Gleichung stark an, umso größer die Systeme werden. Eine Möglichkeit der Simplifizierung ist hier die Born-Oppenheimer-Näherung (Gleichung 2). 

\begin {equation}
\mathcal{H}\left(r,R\right)=\mathcal{H}_{K}\left(R\right)+\mathcal{H}_{e}\left(r,R\right)
\end {equation}

Sie approximiert aufgrund der vielfach höheren Masse der Kerne im Vergleich zu den Elektronen, dass die Bewegungen der Kerne und die der Elektronen separierbar sind. Außerdem wird angenommen, dass die kinetische Energie der Kerne gleich Null und somit die potentielle Energie konstant ist.
Hieraus ergibt sich die elektronische Schrödingergleichung (Gleichung 3) mit dem Hamilton-Operator wobei die Energie nur noch parametrisch von den Koordinaten der Kerne abhängig ist.

\begin {equation}
\mathcal{H}_e\left(r,R\right)\psi\left(r,R\right)=\left(E\left(R\right)-\mathcal{V}_{KK}\left(R\right)\right)\psi_e\left(r,R\right)
\end {equation}

 Somit ergeben sich Wertepaare aus der Energie des Systems und dem Abstand der Kerne, welche die Potentialenergieflächen bilden.
Durch die Born-Oppenheimer-Näherung spielen Isotopeneffekte keine Rolle, da die Kerne als in Ruhe approximiert werden und somit keine kinetische Energie besitzen.
Die Potentialenergieflächen sind abhängig von dem Zusatnd des Moleküls, da die Schrödingergleichung über die Wellenfunktion mit den Quantenzahlen verbunden ist. In der Schwingungsspektroskopie betrachtet man lediglich den Grundzustand.


\end{document}